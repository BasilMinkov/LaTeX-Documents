%%%%%%%%%%%%%%%%%%%%%%%%%%%%%%%%%%%%%%%%%
% Lachaise Assignment
% LaTeX Template
% Version 1.0 (26/6/2018)
%
% This template originates from:
% http://www.LaTeXTemplates.com
%
% Authors:
% Marion Lachaise & François Févotte
% Vel (vel@LaTeXTemplates.com)
%
% License:
% CC BY-NC-SA 3.0 (http://creativecommons.org/licenses/by-nc-sa/3.0/)
% 
%%%%%%%%%%%%%%%%%%%%%%%%%%%%%%%%%%%%%%%%%

%----------------------------------------------------------------------------------------
%	PACKAGES AND OTHER DOCUMENT CONFIGURATIONS
%----------------------------------------------------------------------------------------

\documentclass[12pt,a4paper,oneside]{article}

\usepackage[utf8]{inputenc}
\usepackage[russian]{babel}
\usepackage{graphicx}
\usepackage{wrapfig}
\usepackage{float}
\usepackage{subcaption}
\usepackage{xcolor}
\definecolor{mygray}{gray}{0.4}
\usepackage[top=2.5cm,bottom=2.5cm,right=2.5cm,left=2.5cm,bindingoffset=0cm]{geometry}
\usepackage[colorlinks=true, a4paper=true, pdfstartview=FitV,
linkcolor=black, citecolor=mygray, urlcolor=mygray]{hyperref}
% \usepackage{natbib}


\input{structure.tex} % Include the file specifying the document structure and custom commands
\graphicspath{{images/}}

%----------------------------------------------------------------------------------------
%	ASSIGNMENT INFORMATION
%----------------------------------------------------------------------------------------

\title{Роль пуринэргической системы и L-лактата в регуляции сна: Отчет \#1} % Title of the assignment

\author{Василий Минков\\ \texttt{proveyourselfmail@gmail.com}} % Author name and email address

\date{ФКН НИУ ВШЭ --- \today} % University, school and/or department name(s) and a date

%----------------------------------------------------------------------------------------

\begin{document}

\maketitle % Print the title

%----------------------------------------------------------------------------------------
%	INTRODUCTION
%----------------------------------------------------------------------------------------

\tableofcontents

\section{Аннотация}

В отчете представлены результаты анализа данных, полученных во время записи показаний электроэнцефалографических электродов, акселерометров и термометра, размещенных на теле хомяка. Хомяк в течении 23 дней находился в состоянии торпора (пониженной скорости обмена веществ), позволяющего животному переживать периоды ограниченного количества пищи. Особый интерес представляет динамика изменения энцефалограммы хомяка на протяжении этого времени.

\newpage

\section{Введение}

\subsection{Электроэнцефалография}

\textbf{Электроэнцефалография (ЭЭГ)} -- это метод регистрации электрической активности мозга. ЭЭГ измеряет суммарные колебания напряжения, возникающего в результате ионного тока в мембранах пирамидальных нейронов головного мозга. Пирамидальные нейроны расположенны в извилинах коры больших полушарий. Их отростки перпендикулярны поверхности черепа. Обычно этот метод используется как неинвазивный: электроды располагаются вдоль кожи головы, не нарушая целостности эпителиальной ткани \cite{Luck2005}. Однако используются и инвазивные электроды, помещающиеся непосредственно на поверхность головного мозга. В таком случае метод также часто называют \textbf{электрокортикографией (ЭКоГ)}. На хомяке, ЭЭГ которого была получена в ходе этого эксперимента, были установлены инвазивные электроды.

\subsection{Медленный сон и торпор у хомяков}

Сон традиционно разделяют на пять стадий. Этапы сна с 1-ого по 3-ий называют \textbf{медленным сном} (англ. \textit{Non-rapid eye movement sleep (NREM)}). Медленный сон, испытываемый всеми млекопитающими, характеризуется приостановкой активного контакта организма с окружающей средой и уменьшением расхода энергии по сравнению с бодрствованием. В отличие от \textbf{быстрого сна} (англ. \textit{Rapid eye movement sleep (REM)}), на этих этапах движения глаз практически отсутствуют, сновидения редки, а мышцы не парализованы \cite{McCarley2007}. 

\textbf{Торпор} (англ. \textit{Torpor}) также характеризуется постепенным физиологическим торможением, приводящим к значительному снижению уровня метаболизма, что позволяет животным переживать периоды с ограниченным количеством питательных веществ. Более того, имеются сходства в ЭЭГ между условиями медленного сна и торпора. Однако, в отличии от сна, во время торпора снижается температура тела животного. Более того, у многих животных не обнаруживается состояние торпора, а если и обнаруживается, то может сильно варьироваться степень снижения температуры и скорости метаболизма. Таким образом существуют основания предполагать, что торпор либо является родственным сну состоянием, либо его эволюционным расширением, развившемся у некоторых видов животных \cite{Silvani2018}. 

Прежде чем войти в состояние торпора, температура мозга животного начинает снижаться. У хомяков, по мере снижения температуры коры, медленные ЭЭГ-волны возникают на более низких частотах. Смещение медленных волн в частотной области к меньшим значениям во время торпора представляет функциональный аналог недосыпания, так как аналогичный феномен возникает при депривации сна \cite{Silvani2018}.

\subsection{Восстановление от торпора у хомяков}

В этом эксперименте изучается электрофизиологическая активность хомяка, находящегося в условиях, принуждающих его перейти к торпору. Логично предположить, что в силу депривации сна, возникающей во время торпора, хомяки должны время от времени покидать это состояние. Такие феномены, их отличия от торпора и обычного состояния бодрствования, представляют особый интерес.

\newpage

\section{Метод} % Unnumbered section

Эксперимент над хомяком продолжался приблизительно 23 дня. Хомяк находился в камере с кормом, освещением и необходимым для постройки норы материалом. В течении первого дня яркость света постепенно снижалась, температура опускалась с десяти до четырех градусов Цельсия. Все оставшееся время она находилась на уровне четырех градусов. Раз в день яркость света повышалась на два часа. По истечении 23 дней хомяк умер от истощения. 

На протяжении этого времени проводилась запись трех физиологических показателей животного: электроэнцефалографической (ЭЭГ) активности мозга, записанной с двух электродов; ускорения, записанного с трех акселерометров, соответствующих трем измерениям; и температуры тела. Измерение температуры происходило каждые 10 минут. Запись ЭЭГ и ускорения производились с частотой 250 Гц. 

Запись изменения температуры тела была непрерывной на протяжении всего времени эксперимента. В данных имелась информация о времени сделанного измерения с точностью до минуты. ЭЭГ и показания акселерометра были разбиты на 7 записей с перерывами в несколько часов. В данных имелась информация о времени начала записи с точностью до секунды. При анализе были использованы показания одного ЭЭГ электрода, так как второй записал случайный шум и сетевые наводки, не соответствующие физиологической активности. Также показания только одного акселерометра были использованы, потому что интерес представляло не направление ускорения, а динамика. 

\newpage

\section{Результаты} % Unnumbered section

\subsection{Сопоставление данных во времени}

Первая решенная задача заключалась в сопоставлении показателей ЭЭГ электрода, акселерометра и термометра во времени. Для этого была подсчитана скорость $v_{x}(t_0, T)$ и средняя мощность ЭЭГ сигнала $P(t_0, T)$ на временном промежутке в 10 минут для каждой из семи записей ЭЭГ и акселерометра. Также средняя мощность была получена для сигнала, предварительно отфильтрованного фильтром с конечной импульсной характеристикой (КИХ-фильтр) для частотной полосы 1 -- 4 Гц $P(t_0, T)_{1-4Hz}$ и для частотной полосы 4 -- 8 Гц $P(t_0, T)_{4-8Hz}$. 

\begin{figure}[H]
\centering
\begin{minipage}{.5\textwidth}
  \centering
  \includegraphics[width=\textwidth]{general.png}
  \captionof{figure}{Участок с большим содержанием артефактов не удален.}\label{fig:general}
\end{minipage}%
\begin{minipage}{.5\textwidth}
  \centering
  \includegraphics[width=\textwidth]{general2.png}
  \captionof{figure}{Участок с большим содержанием артефактов удален.}\label{fig:general2}
\end{minipage}
\end{figure}

Скорость на промежутке длинной $T$ может быть подсчитано по формуле $v_{x}(t_0, T) = \int_{t_0}^{t_0 + T} | a_{x}(t)| \; \text{d}t$. В случае физических сигналов средняя мощность на таком же промежутке подсчитывается по формуле $P(t_0, T) = \frac{1}{T} \int_{t_0}^{t_0 + T} [x(t)]^2 \; \text{d}t$. Для фильтрации был выбран КХИ-фильтр, потому что он устойчив, то есть поведение отфильтрованного сигнала «не сильно отличается» от поведения исходного сигнала. Это представляется важным, потому что артефакты, вызванные активностью, отличной от ЭЭГ, могут привышать ЭЭГ по амплитуде и исказить результат работы фильтров других типов.

В результате для каждой из записей были получены векторы значений скорости и мощности на 10-минутных временных интервалах, которые можно было сопоставить значениям термометра. Показания термометра, для которых не было соответствующих значений мощности и ускорения, были удалены из данных (\ref{fig:general}). В записи с 11 по 14 день наблюдается высокий всплеск мощности ЭЭГ и скорости перемещения животного. Такой сигнал сложно интерпретировать и, по всему видимому, он является продолжительным артефактом, который был удален из записи (\ref{fig:general2}, \ref{fig:general4},\ref{fig:general4}). 

\begin{figure}[H]
\centering
\begin{minipage}{.5\textwidth}
  \centering
  \includegraphics[width=\textwidth]{general3.png}
  \captionof{figure}{}\label{fig:general3}
\end{minipage}%
\begin{minipage}{.5\textwidth}
  \centering
  \includegraphics[width=\textwidth]{general4.png}
  \captionof{figure}{}\label{fig:general4}
\end{minipage}
\end{figure}

Судя по графикам, в первом промежутке выхода из торпора значения ЭЭГ были гораздо мощнее, чем при бодрствовании в самом начале эксперимента. Это может значить, что во время выхода из торпора затрата энергии начинается по всему телу для поднятия температуры, что влияет на ЭЭГ. При учете того, что в конце эксперимента хомяк погиб, можно предположить, что затрата энергии при выходе из торпора привела к истощению хомяка. 

\subsection{Статистика}

\begin{figure}[H]
\centering
\begin{minipage}{.5\textwidth}
  \centering
  \includegraphics[width=\textwidth]{t_vs_speed.png}
  \captionof{figure}{}\label{fig:t_vs_speed}
\end{minipage}%
\begin{minipage}{.5\textwidth}
  \centering
  \includegraphics[width=\textwidth]{t_vs_p.png}
  \captionof{figure}{}\label{fig:t_vs_p}
\end{minipage}
\end{figure}

Диаграммы рассеяния позволяют выделить четыре состояния в которых мог находиться хомяк: норма, торпор, переход к торпору и переход к норме. 

Судя по диаграмме рассеяния для температуры тела хомяка и скорости его передвижения (\ref{fig:t_vs_speed}), скорость была ниже при низких температурах тела, что логично, потому что хомяк в этом случае впадал в торпор и не передвигался. При высоких температурах скорость наоборот была больше, потому что хомяк чаще бодрствовал и больше передвигался. При переходе от нормы к торпору видно, что раз хомяк останавливался и засыпал, а после этого у него начиналось понижение температуры тела. При переходе от торпора к норме видно, что скорость увеличивалась постепенно, а значит хомяк постепенно разогревался и был способен к более активному перемещению. 

Диаграмме рассеяния для температуры тела хомяка и мощности ЭЭГ в диапозоне от 4 до 8 Гц (\ref{fig:t_vs_p}) показывает, что мощность была ниже при низких температурах тела. Это можно объясняется экономией энергии в состоянии торпора. При высоких температурах мощность наоборот была больше, потому что хомяк бодрствовал и его мозг расходовал энергию, что приводило к электрической активности клеток. Однако, в отличии от скорости, мощность постепенно росла при выходе в состояние торпора или убывала при входе из состояния торпора. 

\begin{figure}[H]
\centering
\begin{minipage}{.5\textwidth}
  \centering
  \includegraphics[width=\textwidth]{violin_plot1.png}
  \captionof{figure}{}\label{fig:violin_plot1}
\end{minipage}%
\begin{minipage}{.5\textwidth}
  \centering
  \includegraphics[width=\textwidth]{violin_plot2.png}
  \captionof{figure}{}\label{fig:violin_plot2}
\end{minipage}
\end{figure}

В зависимости от того, в каком состоянии находился хомяк, торпора или бодрствования, различия оказались значимы и между скоростью передвижения хомяка ($T = 9435$, $p < 10^{-10}$)(\ref{fig:violin_plot1}), и между мощностью ЭЭГ-сигнала в частотной полосе от 4 до 8 Гц ($T = 620$, $p < 10^{-10}$) (\ref{fig:violin_plot2}). 

\begin{figure}[H]
\centering
\includegraphics[width=0.7\linewidth]{moving_correlation.png}
\caption{}\label{fig:moving_correlation}
\end{figure}

Метод скользящего окна для корреляции Пирсона позволил установить наличие значимой корреляции логарифмированной мощности и температуры тела на разных временных интервалах проведенного эксперимента. Подробный результат представлен на рисунке \ref{fig:moving_correlation}.

\section{Приложение}

Код, выполняющий все вычисления, может быть найден \href{https://github.com/BasilMinkov/Jupyter-Notebooks/blob/master/HamsterEEG.ipynb}{в моем репозитории GitHub}. Код написан в \textbf{Jupyter Notebook} на языке \textbf{Python 3}.


\bibliographystyle{apalike}
\bibliography{/Users/wassilyminkow/Scripts/LaTeX/library.bib}

\end{document}
